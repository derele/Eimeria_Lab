% Options for packages loaded elsewhere
\PassOptionsToPackage{unicode}{hyperref}
\PassOptionsToPackage{hyphens}{url}
%
\documentclass[
]{article}
\usepackage{amsmath,amssymb}
\usepackage{lmodern}
\usepackage{ifxetex,ifluatex}
\ifnum 0\ifxetex 1\fi\ifluatex 1\fi=0 % if pdftex
  \usepackage[T1]{fontenc}
  \usepackage[utf8]{inputenc}
  \usepackage{textcomp} % provide euro and other symbols
\else % if luatex or xetex
  \usepackage{unicode-math}
  \defaultfontfeatures{Scale=MatchLowercase}
  \defaultfontfeatures[\rmfamily]{Ligatures=TeX,Scale=1}
\fi
% Use upquote if available, for straight quotes in verbatim environments
\IfFileExists{upquote.sty}{\usepackage{upquote}}{}
\IfFileExists{microtype.sty}{% use microtype if available
  \usepackage[]{microtype}
  \UseMicrotypeSet[protrusion]{basicmath} % disable protrusion for tt fonts
}{}
\makeatletter
\@ifundefined{KOMAClassName}{% if non-KOMA class
  \IfFileExists{parskip.sty}{%
    \usepackage{parskip}
  }{% else
    \setlength{\parindent}{0pt}
    \setlength{\parskip}{6pt plus 2pt minus 1pt}}
}{% if KOMA class
  \KOMAoptions{parskip=half}}
\makeatother
\usepackage{xcolor}
\IfFileExists{xurl.sty}{\usepackage{xurl}}{} % add URL line breaks if available
\IfFileExists{bookmark.sty}{\usepackage{bookmark}}{\usepackage{hyperref}}
\hypersetup{
  pdftitle={qPCR\_for\_faeces\_protocol},
  pdfauthor={Fay},
  hidelinks,
  pdfcreator={LaTeX via pandoc}}
\urlstyle{same} % disable monospaced font for URLs
\usepackage[margin=1in]{geometry}
\usepackage{graphicx}
\makeatletter
\def\maxwidth{\ifdim\Gin@nat@width>\linewidth\linewidth\else\Gin@nat@width\fi}
\def\maxheight{\ifdim\Gin@nat@height>\textheight\textheight\else\Gin@nat@height\fi}
\makeatother
% Scale images if necessary, so that they will not overflow the page
% margins by default, and it is still possible to overwrite the defaults
% using explicit options in \includegraphics[width, height, ...]{}
\setkeys{Gin}{width=\maxwidth,height=\maxheight,keepaspectratio}
% Set default figure placement to htbp
\makeatletter
\def\fps@figure{htbp}
\makeatother
\setlength{\emergencystretch}{3em} % prevent overfull lines
\providecommand{\tightlist}{%
  \setlength{\itemsep}{0pt}\setlength{\parskip}{0pt}}
\setcounter{secnumdepth}{-\maxdimen} % remove section numbering
\ifluatex
  \usepackage{selnolig}  % disable illegal ligatures
\fi

\title{qPCR\_for\_faeces\_protocol}
\author{Fay}
\date{23/01/2022}

\begin{document}
\maketitle

\hypertarget{quantitative-real-time-pcr-for-faecal-samples}{%
\section{Quantitative real-time PCR for faecal
samples}\label{quantitative-real-time-pcr-for-faecal-samples}}

\hypertarget{measuring-amount-of-dna-with-nanodrop}{%
\subsubsection{Measuring amount of DNA with
NanoDrop}\label{measuring-amount-of-dna-with-nanodrop}}

\begin{itemize}
\tightlist
\item
  Switch on the PC next to nanodrop
\item
  Log in the program of NanoDrop
\item
  Add 1 µL of the solution, in which your DNA is diluted and blank
\item
  Write the sample name
\item
  Add 1 µL of your sample and measure
\item
  Save them on a usb stick
\end{itemize}

\hypertarget{standardizing-the-samples}{%
\subsubsection{Standardizing the
samples:}\label{standardizing-the-samples}}

\begin{itemize}
\tightlist
\item
  Easy way: Use this webbsite to calculate the sample dilutions **
  \url{http://www.desiquintans.com/dilutioncalc}
\item
  Add the desired final volume
\item
  In this case: 50 ng of DNA
\item
  10 ng per µL (5 µL of sample = 50 ng of DNA)
\end{itemize}

\hypertarget{eimeria-specific-primers}{%
\subsection{Eimeria specific primers:}\label{eimeria-specific-primers}}

Prepare the primers and make aliquots

Prepare the layout of your plate

\begin{enumerate}
\def\labelenumi{\arabic{enumi}.}
\tightlist
\item
  Eim\_COI\_qX\_F 5'-TGTCTATTCACTTGGGCTATTGT-3'
\item
  Eim\_COI\_qX\_R 5'-GGA TCACCGTTAAATGAGGCA-3'
\end{enumerate}

\hypertarget{each-reaction-contains-20-uxb5l}{%
\subsubsection{Each reaction contains: (20
µL)}\label{each-reaction-contains-20-uxb5l}}

\begin{itemize}
\tightlist
\item
  1X iTaqTM Universal SYBRⓇ Green Supermix (Bio-Rad, USA)
\item
  400 nM forward and reverse primers
\item
  50 ng template gDNA
\end{itemize}

For 3 repeats (calculate 0.5 extra) 1. STBR: 10 µL x 3.5 = 35 µL 2.
Forward primer: 0.8 µL x 3.5 = 2.8 µL 3. Reverse primer: 0.8 µL x 3.5 =
2.8 µL 4. DNA: 5 µL x 3.5 = 17.5 µL 5. H20: 3.4 µL x 3.5 = 11.9 µL

\hypertarget{pcr-machine}{%
\subsubsection{PCR machine:}\label{pcr-machine}}

\begin{itemize}
\tightlist
\item
  either in the ABI7300 Real-Time PCR system (Applied Biosystems, USA)
\item
  or in a MasterCycler® RealPlex2 (Eppendorf, Germany)
\item
  or in in the CFX96TM Touch System (Bio-Rad, USA)
\end{itemize}

\hypertarget{material-used}{%
\subsubsection{Material used:}\label{material-used}}

DNAs from mock samples, floated oocysts and faecal DNA derived from the
infection experiment

\hypertarget{cycling-conditions}{%
\subsubsection{Cycling conditions:}\label{cycling-conditions}}

Arrange the plate begore in Thermofischer.com Design Analysis new
--\textgreater{} set up pkate --\textgreater{} comparative cd - SYBR

\begin{enumerate}
\def\labelenumi{\arabic{enumi}.}
\tightlist
\item
  initial denaturation at \textbf{95°C} for \textbf{2 min}
\item
  \textbf{40 cycles} of denaturation at \textbf{95°C} for \textbf{15 s}
\item
  annealing at \textbf{55 °C} for \textbf{15 s}
\item
  and extension at \textbf{68°} for \textbf{20 s}
\end{enumerate}

(with data collection at the end of each cycle)

\hypertarget{melting-curve-anaylsis}{%
\subsubsection{Melting curve anaylsis:}\label{melting-curve-anaylsis}}

Melting curve analysis was included to discard primer dimer formation
and non-specific amplification: after the last amplification cycle
temperature was increased from 65 °C to 95 °C with 0.5 °C increments at
3 s/step

\hypertarget{positioning-on-the-plate}{%
\subsubsection{Positioning on the
plate}\label{positioning-on-the-plate}}

Amplifications were performed by triplicate and each run included a
non-template control (NTC).

\hypertarget{analysis-and-interpretation}{%
\subsubsection{Analysis and
interpretation}\label{analysis-and-interpretation}}

Samples with all three replicates showing Tm in the range 74.1 ± 1.78 °C
(observed on positive controls, Additional file 2: Figure S1) were
labelled as ``qPCR positive'', samples with only one or two of the
triplicates showing peaks were designated as negative samples.

For qPCR negative samples, genome copies per gram of faeces were set to
0. ict genome equivalents from Ct.

\end{document}
